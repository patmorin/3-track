\documentclass{patmorin}
\listfiles
\usepackage[utf8]{inputenc}
\usepackage{microtype}
\usepackage{amsthm,amsmath,graphicx}
\usepackage{pat}
\usepackage[letterpaper]{hyperref}
\usepackage[table,dvipsnames]{xcolor}
\definecolor{linkblue}{named}{Blue}
\hypersetup{colorlinks=true, linkcolor=linkblue,  anchorcolor=linkblue,
citecolor=linkblue, filecolor=linkblue, menucolor=linkblue,
urlcolor=linkblue} 
\setlength{\parskip}{1ex}
\usepackage{wasysym}

\title{\MakeUppercase{Track Number 3 Implies Layered Pathwidth 8}}

\author{The Layered $\star$-width Gang}%

%\usepackage[mathlines]{lineno}
%\linenumbers
%\setlength{\linenumbersep}{2.5cm}
%\rightlinenumbers
%\linenumbers
%\newcommand*\patchAmsMathEnvironmentForLineno[1]{%
%  \expandafter\let\csname old#1\expandafter\endcsname\csname #1\endcsname
%  \expandafter\let\csname oldend#1\expandafter\endcsname\csname end#1\endcsname
%  \renewenvironment{#1}%
%     {\linenomath\csname old#1\endcsname}%
%     {\csname oldend#1\endcsname\endlinenomath}}% 
%\newcommand*\patchBothAmsMathEnvironmentsForLineno[1]{%
%  \patchAmsMathEnvironmentForLineno{#1}%
%  \patchAmsMathEnvironmentForLineno{#1*}}%
%\AtBeginDocument{%
%\patchBothAmsMathEnvironmentsForLineno{equation}%
%\patchBothAmsMathEnvironmentsForLineno{align}%
%\patchBothAmsMathEnvironmentsForLineno{flalign}%
%\patchBothAmsMathEnvironmentsForLineno{alignat}%
%\patchBothAmsMathEnvironmentsForLineno{gather}%
%\patchBothAmsMathEnvironmentsForLineno{multline}%
%}


\newcommand{\question}[1]{\textbf{\color{red}Question:}~#1}

\DeclareMathOperator{\ob}{obs}
\DeclareMathOperator{\planeobs}{plane-obs}

\newcommand{\eps}{\epsilon}

\DeclareMathOperator{\tr}{tr}
\DeclareMathOperator{\pw}{pw}
\DeclareMathOperator{\lpw}{lpw}
\DeclareMathOperator{\wlpw}{wlpw}

\DeclareSymbolFont{sfoperators}{OT1}{cmss}{m}{n}
\DeclareSymbolFontAlphabet{\mathsf}{sfoperators}

\makeatletter
\def\operator@font{\mathgroup\symsfoperators}
\makeatother

%\pagenumbering{roman}
\begin{document}
%\begin{titlepage}
\maketitle
%
\begin{abstract}
  We answer a question of Bannister \etal\ (2015) by showing that graphs with
  track number 3 have layered pathwidth at most 8.
\end{abstract}
%\end{titlepage}
%
%\tableofcontents
%
%\newpage


\section{Introduction}
\pagenumbering{arabic}

A \emph{$t$-track assignment} of a graph $G$ is a partition of $V(G)$
into $t$ independent sets $T_1,\ldots,T_t$ along with $t$ relations
$\prec_1,\ldots,\prec_t$, where $\prec_i$ defines a total order on $T_i$
for each $i\in\{1,\ldots,t\}$. An \emph{X-crossing} in a track layout is a
pair of edges $vw$ and $xy$ such that, for some $i,j\in\{1,\ldots,t\}$,
$v,x\in T_i$ with $v\prec_i x$ and $w,y\in T_j$ with $y\prec_j w$.
A $t$-track assignment is called a \emph{$t$-track layout} if it has no
X-crossing.  The minimum number of tracks in any $t$-track layout of $G$
is called the track number of $G$ and is denoted as $\tr(G)$.

A \emph{path decomposition} of $G$ is a sequence of sets
$B_1,\ldots,B_p\subseteq V(G)$ called \emph{bags} such that
\begin{enumerate}
   \item For each $v\in V(G)$, the bags that contain $v$ form a contiguous subsequence.
   \item For each edge $vw\in E(G)$, there is some bag that contains both $v$ and $w$.
\end{enumerate}
The \emph{width} of a path decomposition is the size of its largest bag.  The \emph{pathwidth} of $G$, denoted $\pw(G)$, is the minimum width of any path decomposition of $G$.

A \emph{layering} of $G$ is a mapping $\ell:V(G)\to\Z$ with the
property that $vw\in E(G)$ implies $|\ell(u)-\ell(w)|\le 1$. One
can also think of a layering as a partition of $G$'s vertices into
sets indexed by integers, where the \emph{layer} $L_i=\{v\in V(G)
: \ell(v)=i\}$.  A \emph{layered path decomposition} of $G$ consists
of a layering $\ell$ and a path decomposition $B_1,\ldots,B_p$ of $G$.
The \emph{(layered) width} of a layered path decomposition is the maximum
size of the intersection of a bag and a layer, i.e., $\max\{|L_i\cap
B_j|:i\in\Z,\, j\in\{1,\ldots,p\}\}$.  The \emph{layered pathwidth} of
$G$, denoted $\lpw(G)$ is the smallest (layered) width of any layered
path decomposition of $G$.

In this paper, we prove the following result, resolving an open question
posed by Bannister \etal\ \cite{bannister.devanny.ea:track}:

\begin{thm}\thmlabel{main}
  If $\tr(G)\le 3$, then $\lpw(G)\le 8$.
\end{thm}

\section{Proof of \thmref{main}}

It will be slightly easier to prove our result for a weaker notion
of layering.  An \emph{$s$-weak layering} of $G$ is a mapping
$\ell:V(G)\to\Z$ with the property that, for every $vw\in E(G)$,
$|\ell(v)-\ell(w)|\le s$.  The sets $L_i=\{v\in V(G): \ell(v)=i\}$
are called $s$-weak layers.  The terms \emph{$s$-weak layered path
decomposition} and \emph{$s$-weak layered pathwidth} of $G$, denoted
$\lpw_s(G)$, are defined the same way as layered path decompositions
and layered pathwidth, but with respect to $s$-weak layerings of $G$.

\begin{lem}\lemlabel{weak}
  For any $s\in\N$, $\lpw(G) \le s\cdot\lpw_s(G)$.
\end{lem}

\begin{proof}
  Given an $s$-weak layered decomposition of $G$ with $s$-weak layering $\ell$, we define
  $\ell'(v)=\lfloor\ell(v)/s\rfloor$. Now $\ell'$ is clearly a layering of $G$ and, by definition, for any bag $B_j$ and any $s$-weak layer $L_i=\{v\in V(G):\ell(v)=i\}$, $|L_i\cap B_j|\le \lpw_s(G)$.  Therefore, since
  any layer $L_i'=\{v\in V(G):\ell'(v)=i\}$ is the union of $s$ weak layers,
  $|L_i'\cap B_j|\le s\cdot\lpw_s(G)$ for all $i\in Z$.
\end{proof}

Let $G$ be an edge maximal $n$-vertex graph with $\tr(G)=3$.  It is
helpful to recall that $G$ is a planar graph that has a straight-line
plane drawing with the vertices of $T_1$ placed on the positive x-axis,
the vertices of $T_2$ placed on the positive y-axis and the vertices of
$T_3$ placed on the ray $\{(a,a):a<0\}$. See \figref{planar-view}. (The
edges in \figref{planar-view} are drawn as curves only to help
readability, . The drawing would be non-crossing even if all edges were
drawn as straight lines.)

\begin{figure}
  \begin{center}
     \includegraphics{figs/path-1}
  \end{center}
  \caption{The standard planar embedding of a 3-track graph.}
  \figlabel{planar-view}
\end{figure}

Let $T_1,\ldots,T_3$ be a 3-track layout of $G$ with
$T_1=\{x_1,\ldots,x_{n_1}\}$, $T_2=\{y_1,\ldots,y_{n_2}\}$, and
$T_3=\{z_1,\ldots,z_{n_3}\}$ and the total orders $\prec_1,\ldots,\prec_3$
are implicit so that, for example $z_i\prec_3 z_j$ if and only if $i<j$.
In terms of \figref{planar-view}, this means that $x_1,y_1,z_1$ form
the triangular face containing the origin and $x_{n_1},y_{n_2},z_{n_3}$
form the cycle on the boundary of the outer face.
From this point onward, all track indices are implicitly taken ``modulo 3''
so that for any integer $i$, $T_i$ refers to the track $T_{i'}$ with
index $i'=((i-1)\bmod 3)+1$.  \thmref{main} is a consequence of the
following lemma.

\begin{lem}\lemlabel{main}
  The graph $G$ described above has a 4-weak layered path decomposition
  $\ell$, $B_1,\ldots,B_p$ of (layered) width $2$ in which 
  \begin{enumerate}
    \item for each $i\in\{1,\ldots,3\}$ and each $v\in T_i$,
      $\ell(v)\equiv i\pmod 3$;
    \item $B_1=\{x_1,y_1,z_1\}$;
    \item $\ell(x_1)=1$, $\ell(y_1)=2$, and $\ell(z_1)=3$;
    \item $B_p=\{x_{n_1},y_{n_2},z_{n_3}\}$; and
    \item $x_{n_1},y_{n_2},z_{n_3}$ appear in 3 consecutive layers.
  \end{enumerate}
\end{lem}

Before proving \lemref{main}, we show how it implies \thmref{main}.
Since layered pathwidth is monotone with respect to the addition of edges,
it is safe to assume (as \lemref{main} does) that $G$ is edge maximal.
By \lemref{main}, therefore $G$ has $\lpw_4(G)=2$ and therefore, by
\lemref{weak}, $\lpw(G)\le 8$.

\begin{proof}[Proof of \lemref{main}]
  Our proof is by induction on the number of vertices of $G$.  If
  $|V(G)|\le 4$, then the result is trivial.  Next, Suppose that $G$ has
  a cut set $C=\{x_i,y_j,z_k\}$ having exactly one vertex in each track.
  Since $G$ is edge maximal, $x_i,y_j,z_k$ form a cycle in $G$.  Now,
  the subgraph $G_1$ of $G$ induced by $\{x_1,\ldots,x_i, y_1,\ldots,y_j,
  z_1,\ldots,z_k\}$ is an edge maximal graph with $\tr(G_1)=3$ and we
  can inductively apply \lemref{main} to find a width-2 4-weak layered
  path decomposition of $G_1$ in which $x_i,y_j,z_k$ are in the last bag
  and are assigned to three consecutive layers $r+1$, $r+2$, and $r+3$.
  Note that there are three possible assignments of $x_i,y_j,z_k$ to
  these three layers depending on the value of $r\bmod 3$.  Suppose,
  without loss of generality, that $\ell(y_j)=r+1$ (so $\ell(z_k)=r+2$
  and $\ell(x_i)=r+3$.)

  Next, consider the graph $G_2$ induced by
  $\{x_i,\ldots,x_{n_1},y_j,\ldots,y_{n_2},z_k,\ldots,z_{n_3}\}$.
  We apply \lemref{main} inductively on $G_2$ relabelling tracks to
  ensure that in the resulting layered decomposition $\ell(y_j)=1$,
  $\ell(z_k)=2$ and $\ell(x_i)=3$.   We can now obtain a width-2 4-weak
  layered path decomposition of $G$ by joining the two decompositions.
  In particular,  concatenating the sequence of bags for $G_1$ with
  the sequence of bags for $G_2$ gives a path decomposition of $G$
  and addding $r$ to the indices of all layers in the layering of $G_2$
  gives a 4-weak layering of $G$.

  Thus, all that remains is to study the case where $G$ contains no cut
  set having exactly one vertex on each track.  We claim that, in this
  case, $G$ contains the edge $x_1z_2$ or it contains the edge $z_1x_2$.
  Since $G$ is edge-maximal, this is obvious unless $n_1=n_3=1$ so
  that neither $z_2$ nor $x_2$ exist.  However, since $|V(G)|\ge 5$,
  this would imply that $x_1,z_1,y_2$ is a cut set with one vertex on
  each track, since its removal separates $y_1$ from $y_3$.

  Assume, without loss of generality, that $G$ contains the edge $z_1x_2$.
  We will construct a path $P$
  starting with the the subpath $x_1,y_1,z_1,x_2$ as follows: When $P$
  has reached a vertex $v\in T_i$, it continues to the lowest indexed
  vertex in $T_{i+1}$ that is adjacent to $v$ but not already in $P$. This
  process terminates when $v$ has no neighbour in $T_{i+1}$ that is not
  already a part of $P$.  An example is illustrated in \figref{path}.

  \begin{figure}
     \begin{center}
        \includegraphics{figs/path-2}
     \end{center}
     \caption{The path $P$ used in the proof of \lemref{main}.}
     \figlabel{path}
  \end{figure}

  Let $P=v_1,\ldots,v_r$ be the resulting path.  We claim that,
  for each $t\in \{1,\ldots,3\}$, the subsequence of $P$ induced by
  $T_t$ is increasing. This can be proven by induction the number of
  vertices added to $P$. Refer to \figref{combo}(a).  If $v_a=x_i$,
  then $v_{a-3}=x_{i'}$ with (by induction) $i'< i$.  Now, $P$ contains
  an edge $v_{a-3}v_{a-2}=x_{i'}y_{j'}$ for some $j'$.  The existence of
  this edge in $G$ ensures that the next vertex $v_{a+1}$ (if any) added
  to $P$ is $y_j$ for some $j > j'$.  (Using a vertex $y_j$ with $j <
  j'$ would imply that $x_{i'}y_{j'}$ and $x_iy_j$ form an X-crossing.)

  \begin{figure}
    \begin{center}
       \begin{tabular}{cc}
         \includegraphics{figs/monotone} & \includegraphics{figs/closed} \\
         (a) & (b)
       \end{tabular}
    \end{center}
    \caption{Two steps in the proof of \lemref{main}: (a)~the sequence $P\cap T_2$ is monotone and (b)~the $v_{r-2},v_{r-1},v_r$ is a cycle.}
    \figlabel{combo}
  \end{figure}


  Next, we claim that
  $\{v_{r-2},v_{r-1},v_{r}\}=\{x_{n_1},y_{n_2},z_{n_2}\}$. To see this,
  suppose without loss of generality, that $v_r\in T_1$, so $v_r=x_i$
  for some $i\in\{1,\ldots,n_1\}$. Refer to \figref{combo}(b). The only
  reason $P$ ends at $x_i$ is that $x_i$ has no neighbour in $T_{2}$
  that is not already in $P$.  Since $G$ is edge maximal, this already
  implies that $v_{r-2}=y_{n_2}$, otherwise $v_r$ would have a neighbor
  in $T_2$ that is not already in $P$.  Again, since $G$ is edge maximal,
  this implies $v_r$ is adjacent to $v_{r-2}$.

  Thus far, we have established that $v_{r-2},v_{r-1},v_{r}$ forms a
  cycle $C$ and, in the planar embedding of $G$, $C$ separates $v_{r-3}$
  (which is inside of $C$) from any vertices outside of $C$.  If there are
  any vertices outside of $C$, this would make $C$ a cut set consisting
  of one vertex from each track, which we have already ruled out.
  Therefore, there are no vertices outside of $C$ and 
  $C=x_{n_1},y_{n_2},z_{n_2}$, as claimed.

  Thus far, we have established that $P=v_1,\ldots,v_r$ is a path whose
  first three vertices are $x_1,y_1,z_1$ and whose last three vertices
  are $x_{n_1}$, $y_{n_2}$ and $z_{n_3}$ (not necessarily in order).

  We assign layers to the vertices of $P$ as follows: For each
  vertex $v_i$ on $P$, we set $\ell(v_i)=i$.  Note that this satisfies
  Conditions~3 and 5 of the lemma and also satisfies Condition~1 for the
  vertices of $P$.  For each $t\in\{1,\ldots,3\}$, any vertex $v\in T_t$
  that is not in $P$ is assigned to the same layer as $v$'s immediate
  precessor in $P\cap T_t$.  This assignment satisifies Condition~1 for
  vertices not in $P$.  Finally, observe that this gives a 4-weak layering
  of $G$ (in the worst case, a vertex $v_i$ is adjacent to $v_{i+4}$).

  Now, consider the graph $G-P$ obtained by removing the vertices of $P$
  from $G$ (see \figref{g-p}).  We claim that this graph is a levelled
  planar graph \cite{bannister.devanny.ea:track} in which the levels of
  the vertices are given by the layering $L_1,\ldots,L_\ell$.  Refer to
  \figref{prism}.  One way to see this is to imagine $G$ as being drawn
  with its vertices on the three vertical edges of a triangular prism
  so that $x_1,y_1,z_1$ are the vertices of one triangular face and
  $x_{n_1},y_{n_2},z_{n_3}$ are the vertices of the other triangular
  face.  Now, if we remove the top and bottom faces of this prism,
  cut it along the embedding of $P$, and unfold the resulting surface
  so that it lies in the plane, then we obtain a drawing of $G-P$ in
  which the vertices lie on a set of parallel lines and in which the
  edges only join vertices on two consecutive lines.  This gives the
  desired levelled planar drawing of $G-P$.

  \begin{figure}
  \begin{center}
  \begin{tabular}{cc}
  \includegraphics{figs/path-2}
  \includegraphics{figs/path-3}
  \end{tabular}
  \end{center}
  \caption{The graph $G-P$ is a levelled planar graph.}
  \figlabel{g-p}
  \end{figure}


  \begin{figure}
     \begin{center}
        \includegraphics{figs/prism}
     \end{center}
     \caption{Cutting a prism along $P$ to obtain a levelled planar
      drawing of $G-P$.
      In the unfolding, edges of $G-P$ are drawn in black. Edges that
      span 4 layers are drawn in orange.}
     \figlabel{prism}
  \end{figure}

  By a result of Bannister \etal\ \cite[Proof of
  Theorem~5]{bannister.devanny.ea:track}, this graph has a
  layered path decomposition $B_1,\ldots,B_p$ of width 1 using layers
  $L_1,\ldots,L_\ell$.  If we add the vertices of $P$ to every bag
  of this decomposition we obtain a 4-weak layered decomposition of
  $G$ with width 2.  Finally, to satisfy Conditions 2 and 4 of the
  lemma, we prepend a bag $B_0=\{x_1,y_1,z_1\}$ and append a bag
  $B_{p+1}=\{x_{n_1},y_{n_2},z_{n_3}\}$.
\end{proof}

\section{Remarks}

An obvious question is how far \thmref{main} can be generalized. As
Bannister \etal\ \cite{bannister.devanny.ea:track} point out, it can not be generalized to graphs with
track number 4; there exists graphs with track number 4 that have 
unbounded layered pathwidth.

Another generalization, that arises naturally from the standard planar
drawing of 3-track graphs is the following. We say that a graph $G$
is a $k$-web if it has a straight line drawing with its vertices placed
on $k$ concentric rays and such that every edge joins a pair of vertices
on two consecutive rays.  The set of 3-webs is exactly the set of graphs
with track-number at most 3.  Our proof of \lemref{main} generalizes in a straightforward manner to the following result:

\begin{thm}
   Every $k$-web $G$ has layered pathwidth at most $2(k+1)$.
\end{thm}

In our proof of \lemref{main} we use the trick of cutting a prism
along a path $P$ in order to obtain a levelled planar drawing of $G-P$.
Bannister \etal\ \cite{bannister.devanny.ea:track} use what, on the surface
appearss to be a somewhat different method to obtain a levelled planar
drawing of a bipartite 3-track graph.  In particular, they use an argument
based on winding numbers.  These two techniques, however are not so
different. Their winding-number argument shows (though not explicitly)
that the standard drawing of a  bipartite 3-track graph does not contain
any cycle that contains the origin.  This means that if we draw this graph
on triangular prism, then there is a path that from triangular face of
the prism to the other that avoids all edges and vertices of the graph.
Cutting along this path and unfolding gives a levelled planar drawing.


\bibliographystyle{plain}
\bibliography{3-track}

\end{document}


